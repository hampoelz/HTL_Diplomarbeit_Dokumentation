\subsubsection{API - Bridge module}
\label{sec:Capacitor-BrowserView:API_BridgeModule}

This plugin includes a bridge between the Capacitor layer and the loaded web page in the BrowserView(s).
However, this feature is disabled by default. It can be enabled either globally via the plugin configuration or for each BrowserView individually during its creation.

If the bridge feature is enabled, the global object \code[javascript]{window.CapacitorBrowserView} is available on web pages within the corresponding BrowserView.

The \code[javascript]{window.CapacitorBrowserView} object provides an \ac{api} to communicate between the Capacitor layer and the web pages.

It has the following methods to listen for events and send messages:

\begin{itemize}
  \setlength\itemsep{-0.8em}
  \item \code{send(...)}
  \item \code{addListener(string, ...)}
\end{itemize}

% -------------------- %

\paragraph{CapacitorBrowserView.send(...)}

\begin{minted}{typescript}
  send: (
    eventName: string,
    ...args: any[]
  ) => void
\end{minted}

Sends a message to the Capacitor layer via \code{eventName}, along with arguments.
Arguments will be serialized with \ac{json}.

\begin{itemize}
  \setlength\itemsep{-0.8em}
  \item \code{eventName}: The name of the event being send to.
  \item \code{args}: The Array of arguments to send.
\end{itemize}

% -------------------- %

\paragraph{CapacitorBrowserView.addListener(string, ...)}

\begin{minted}{typescript}
  addListener: (
    eventName: string,
    callback: (...args: any[]) => void
  ) => void
\end{minted}

Listens to \code{eventName} and calls \code{callback(args...)} when a new message arrives from the Capacitor layer.

\begin{itemize}
  \setlength\itemsep{-0.8em}
  \item \code{args}: The received array of arguments.
\end{itemize}

% -------------------- %
