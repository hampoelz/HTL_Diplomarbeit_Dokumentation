\subsubsection{Getting Started}
\label{sec:Capacitor-BrowserView:GettingStarted}

This guide shows how to add a BrowserView to a Capacitor application and communicate with the web page.

\paragraph{Minimal example}
\label{sec:Capacitor-BrowserView:MinimalExample}

To embed an additional web page to the Capacitor app, first a new BrowserView needs to be created.
This is done by importing the BrowserView class from the Capacitor-BrowserView plugin and calling the create method.
After that, just set its bounds and load a website.

\begin{minted}{typescript}
import { BrowserView } from 'capacitor-browserview';

// Creates a new BrowserView, sets its bounds and loads a website.
const myView = await BrowserView.create();
myView.setBounds({ bounds: { x: 0, y: 0, width: 300, height: 600 } });
myView.loadUrl({ url: "https://capacitorjs.com/" });
\end{minted}

At this point, the BrowserView should be created within the application and the website should start loading.
Using the returned reference to the BrowserView, it is easy to interact with it, as shown here in a few examples:

\begin{minted}{typescript}
// Reloads the current web page.
myView.reload();

// Gets the user-agent for this web page.
const { userAgent } = await myView.getUserAgent();
console.log("The following UserAgent is used for this web page: "
    + userAgent);

// Handles finished page load.
myView.addListener('did-finish-load', () => {
    console.log("Congratulations! The web page loaded successfully.");
});

// Handles title changes of the web page.
myView.addListener('page-title-updated', event => {
    console.log(`The web page has changed its title to '${event.title}'.`);
});

// Removes the BrowserView from the app, and cleans up references.
myView.destroy();
\end{minted}

\newpage

\paragraph{Inter-Process Communication}
\label{sec:Capacitor-BrowserView:InterprocessCommunication}

This plugin includes a bridge between the Capacitor layer and the loaded web page in the BrowserView(s).
However, this feature is disabled by default. It can be enabled either globally via the plugin configuration or for each BrowserView individually during its creation.

If the bridge feature is enabled, the global object \code[javascript]{window.CapacitorBrowserView} is available on web pages within the corresponding BrowserView.

To use the bridge, the website first needs to be modified so that it is capable of receiving and returning messages.
For example, add the following sample code to the web page:

\begin{minted}{javascript}
// Listens to "msg-from-capacitor" from the Capacitor layer.
CapacitorBrowserView.addListener("msg-from-capacitor", message => {
    console.log('Message from Capacitor-App: ' + message);

    // Sends a message back to the Capacitor layer.
    CapacitorBrowserView.send("msg-from-web",
        `Replying to the message '${message}'.`,
        "And optionally add further args"
    );
});
\end{minted}

After that, a new BrowserView with an enabled bridge needs be set up.
In addition, the Capacitor app should also be able to receive messages and send a message to the website.
Note that the web page must be loaded entirely before it can receive messages.
This could look something like this:

\begin{minted}{typescript}
import { BrowserView } from 'capacitor-browserview';

const myView = await BrowserView.create({ enableBridge: true });
myView.setBounds({ bounds: { x: 0, y: 0, width: 300, height: 600 } });

// Listens to "msg-from-web" from the web page.
myView.addMessageListener('msg-from-web', data => {
    console.log("Message from web page: ", data.args);
});

// Waits for the web page to finish loading.
myView.addListener('did-finish-load', () => {

    // Sends a message to the web page.
    myView.sendMessage({
        eventName: "msg-from-capacitor",
        args: [ "Hello from Capacitor!" ]
    });

});

myView.loadUrl({ url: "https://webpage.dev/" });
\end{minted}

A full \ac{api} documentation can be found in the \hyperref[sec:Capacitor-BrowserView:API_BridgeModule]{API - Bridge module} section.
