\subsection{Benutzung}

In diesem Kapitel wird Schritt für Schritt erklärt, wie das Plugin verwendet wird.
Es enthält Beispiele für einfache und komplexe Projekte sowie entsprechende \acs{api}"=Dokumentationen.

Dieser Teil der Dokumentation bezieht sich auf die Vorabversion \code{1.0.0-beta.3} des Plugins.
Die neueste Version und die dazugehörige Dokumentation sind auf der \href{https://github.com/hampoelz/Capacitor-BrowserView}{GitHub-Projektseite}\footnotemark[0] zu finden.

Um das Plugin für eine globale Zielgruppe zugänglich zu machen, wurde die Nachfolgende Dokumentation in englischer Sprache verfasst.

\footnotetext[0]{\url{https://github.com/hampoelz/Capacitor-BrowserView}}

\vspace{3em}

\textbf{Table of contents}

\begin{itemize}
    \setlength\itemsep{-1em}
    \item \hyperref[sec:Capacitor-BrowserView:GettingStarted]{Getting Started}
    \vspace{\itemsep}
    \begin{itemize}
        \setlength\itemsep{-1em}
        \item \hyperref[sec:Capacitor-BrowserView:MinimalExample]{Minimal example}
        \item \hyperref[sec:Capacitor-BrowserView:InterprocessCommunication]{Inter-Process Communication}
    \end{itemize}
    \item \hyperref[sec:Capacitor-BrowserView:Configuration]{Configuration}
    \item \hyperref[sec:Capacitor-BrowserView:API_BridgeModule]{API - Bridge module}
    \item \hyperref[sec:Capacitor-BrowserView:API_CapacitorLayer]{API - Capacitor layer}
\end{itemize}

\newpage

\input{src/03_Capacitor-BrowserView/04_Benutzung/01_GettingStarted.tex}
\clearpage

\addurlfn{mixed-content}{Mixed content}{https://developer.mozilla.org/en-US/docs/Web/Security/Mixed_content}

\subsubsection{Configuration}
\label{sec:Capacitor-BrowserView:Configuration}

The following configurations are applied globally to all BrowserViews.
However, most of them can be overridden individually when a BrowserView is created or changed later via methods.

These config values are available:

\begin{config}[Capacitor-BrowserView / Configuration]
  \code{url}                  & \code[typescript]{string}   & Default external \ac{url} loaded in BrowserViews. \\ \hline
  \code{allowMultipleWindows} & \code[typescript]{boolean}  & Open links that request a new tab or window in the external browser instead of the BrowserViews. \\
                              &                             & Defaults to \code[typescript]{true} \\ \hline
  \code{allowNavigation}      & \code[typescript]{string[]} & Set regular expressions to which the BrowserViews can navigate additional. By default, all external \acp{url} are opened in the external browser (not the BrowserView). \\
                              &                             & Defaults to \code[typescript]{[]} \\ \hline
  \code{enableBridge}         & \code[typescript]{boolean}  & Enable a bridge between the Capacitor layer and the loaded web page. \\
                              &                             & Defaults to \code[typescript]{false} \\ \hline
  \code{overrideUserAgent}    & \code[typescript]{string}   & Default user-agent for BrowserViews. \\ \hline
  \code{appendUserAgent}      & \code[typescript]{string}   & String to append to the original user-agent for BrowserViews. This is disregarded if \code{overrideUserAgent} is used. \\ \hline
  \code{backgroundColor}      & \code[typescript]{Color}    & Default background color for BrowserViews. \\ \hline
\end{config}

\begin{config}[Capacitor-BrowserView / Configuration (Android)]
  \codebreak{android}{OverrideUserAgent} & \code[typescript]{string}   & Overrides global \code{overrideUserAgent} option on Android. \\ \hline
  \codebreak{android}{AppendUserAgent}   & \code[typescript]{string}   & Overrides global \code{appendUserAgent} option on Android. \\ \hline
  \codebreak{android}{BackgroundColor}   & \code[typescript]{Color}    & Overrides global \code{backgroundColor} option on Android. \\ \hline
  \codebreak{android}{AllowMixedContent} & \code[typescript]{boolean}  & Enable mixed content in the BrowserViews for Android. \fn{mixed-content} is disabled by default for security. During development, this option may need to be enabled to allow the BrowserViews to load files from different schemes. \\
                                         &                             & \textbf{This is not intended for use in production.} \\ 
                                         &                             & Defaults to \code[typescript]{false} \\ \hline
\end{config}

\begin{config}[Capacitor-BrowserView / Configuration (Electron)]
  \codebreak{electron}{OverrideUserAgent} & \code[typescript]{string}   & Overrides global \code{overrideUserAgent} option on Electron. \\ \hline
  \codebreak{electron}{AppendUserAgent}   & \code[typescript]{string}   & Overrides global \code{appendUserAgent} option on Electron. \\ \hline
  \codebreak{electron}{BackgroundColor}   & \code[typescript]{Color}    & Overrides global \code{backgroundColor} option  on Electron. \\ \hline
  \codebreak{electron}{AllowMixedContent} & \code[typescript]{boolean}  & Enable mixed content in the BrowserViews for Electron. \fn{mixed-content} is disabled by default for security. During development, this option may need to be enabled to allow the BrowserViews to load files from different schemes. \\
                                          &                             & \textbf{This is not intended for use in production.} \\ 
                                          &                             & Defaults to \code[typescript]{false} \\ \hline
\end{config}

\newpage

\textbf{Examples}

In \code{capacitor.config.json}:

\begin{minted}{json}
{
  "plugins": {
    "CapacitorBrowserView": {
      "url": "https://capacitorjs.com/",
      "allowMultipleWindows": false,
      "allowNavigation": [
        "capacitorjs\.com",
        "ionic\.io\/blog\/.*capacitor.*"
      ],
      "enableBridge": true,
      "overrideUserAgent": "Mozilla/5.0 (CapacitorJS) CapacitorApp/1.0",
      "appendUserAgent": "CapacitorApp/1.0",
      "backgroundColor": "#ffffff",
      "androidOverrideUserAgent": "Mozilla/5.0 (Android) CapacitorApp/1.0",
      "androidAppendUserAgent": "CapacitorApp/1.0 (Android)",
      "androidBackgroundColor": "#ffffff",
      "androidAllowMixedContent": false,
      "electronOverrideUserAgent": "Mozilla/5.0 (Electron) CapacitorApp/1.0",
      "electronAppendUserAgent": "CapacitorApp/1.0 (Electron)",
      "electronBackgroundColor": "#ffffff",
      "electronAllowMixedContent": false
    }
  }
}
\end{minted}

In \code{capacitor.config.ts}:

\begin{minted}{typescript}
/// <reference types="capacitor-browserview" />

import { CapacitorConfig } from '@capacitor/cli';

const config: CapacitorConfig = {
  plugins: {
    CapacitorBrowserView: {
      url: "https://capacitorjs.com/",
      allowMultipleWindows: false,
      allowNavigation: [
        "capacitorjs\.com",
        "ionic\.io\/blog\/.*capacitor.*"
      ],
      enableBridge: true,
      overrideUserAgent: "Mozilla/5.0 (CapacitorJS) CapacitorApp/1.0",
      appendUserAgent: "CapacitorApp/1.0",
      backgroundColor: "#ffffff",
      androidOverrideUserAgent: "Mozilla/5.0 (Android) CapacitorApp/1.0",
      androidAppendUserAgent: "CapacitorApp/1.0 (Android)",
      androidBackgroundColor: "#ffffff",
      androidAllowMixedContent: false,
      electronOverrideUserAgent: "Mozilla/5.0 (Electron) CapacitorApp/1.0",
      electronAppendUserAgent: "CapacitorApp/1.0 (Electron)",
      electronBackgroundColor: "#ffffff",
      electronAllowMixedContent: false,
    },
  },
};

export default config;
\end{minted}

\printfn

\clearpage

\input{src/03_Capacitor-BrowserView/04_Benutzung/03_APIBridgeModule.tex}
\clearpage

\input{src/03_Capacitor-BrowserView/04_Benutzung/04_APICapacitorLayer.tex}
