\subsubsection{Konfiguration}
\label{sec:WebApp-Wrapper:Konfiguration}

Die folgenden Konfigurationen sind verfügbar:

\begin{configuration}{WebApp-Wrapper / Konfiguration}
  \code{appId}        & \code[typescript]{string}     & Ein eindeutiger Bezeichner für die Anwendung. Sie wird auch als Bundle-ID in iOS und als Anwendungs-ID in Android bezeichnet. Sie muss in umgekehrter Domänennamenschreibweise angegeben werden, was im Allgemeinen einen Domänennamen darstellt, der dem Unternehmen gehört.\\ \hline
  \code{appName}      & \code[typescript]{string}     & Der benutzerfreundliche Name der Anwendung. Dies sollte der Name sein, der im AppStore angezeigt wird. \\ \hline
  \code{appUrl}       & \code[typescript]{string}     & Die Adresse der Webseite, die in der Anwendung angezeigt wird. \\ \hline
  \code{primaryColor} & \code[typescript]{string}     & Die Hauptfarbe der Benutzeroberfläche. Der Standardwert ist \textcolor[HTML]{BB4747}{\code*{"\#1095c1"}} \\ \hline
  \code{menuColor}    & \code[typescript]{string}     & Die Hintergrundfarbe der Menüleiste. \\ \hline
  \code{menu}         & \code[typescript]{MenuItem[]} & Eine Liste von Menüelementen. Jedes angegebene Menüelement wird in der Menüleiste angezeigt. Wenn diese Konfiguration nicht festgelegt wurde, wird keine Menüleiste angezeigt. \\ \hline
  \code{contactUrl}   & \code[typescript]{string}     & Eine Webseite, unter der ein Fehlerbericht erstellt werden kann, wenn die angegebene Webseite aufgrund eines Serverfehlers nicht geladen werden kann. \\ \hline
\end{configuration}

\newpage

\textbf{Beispiele}

In \code{capacitor.config.json}:

\begin{minted}{json}
{
  "appId": "com.company.app",
  "appName": "Company App",
  "appUrl": "https://app.company.com/",
  "primaryColor": "#1b66c9",
  "menuColor": "#202124",
  "menu": [
    {
      "id": "test-button",
      "type": "button",
      "text": "Hello World"
    }
  ],
  "contactUrl": "",
  "webDir": "dist"
}
\end{minted}

\begin{warning}
    Die Konfiguration \code{webDir} darf nicht verändert werden.
    Capacitor verwendet diese Konfiguration, um die Plattformen mit den Projektdateien zu synchronisieren.
    Wenn diese Konfiguration geändert wird, kann es zu Problemen bei der Synchronisierung kommen.
\end{warning}
