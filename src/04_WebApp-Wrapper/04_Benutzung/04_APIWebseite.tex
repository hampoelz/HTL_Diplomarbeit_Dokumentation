\subsubsection{API - Webseite}

Der WebApp-Wrapper bietet auf der Webseite eine \ac{api} an, um Konfigurationen zu ändern und Nachrichten zwischen der Webseite und dem Node.js Projekt auszutauschen.

Die folgenden Methoden stehen zur Verfügung:

\begin{itemize}
  \setlength\itemsep{-0.8em}
  \item \code{CapacitorBrowserView.send('wrapper:updateColor', ...)}
  \item \code{CapacitorBrowserView.send('wrapper:updateMenuColor', ...)}
  \item \code{CapacitorBrowserView.send('wrapper:enableMenu', ...)}
  \item \code{CapacitorBrowserView.send('wrapper:addMenuItem', ...)}
  \item \code{CapacitorBrowserView.send('wrapper:removeMenuItem', ...)}
  \item \code{CapacitorBrowserView.send('wrapper:openMessageScreen', ...)}
  \item \code{CapacitorBrowserView.send('wrapper:closeMessageScreen')}
  \item \code{CapacitorBrowserView.send('wrapper:sendMessage', ...)}
  \item \code{CapacitorBrowserView.addListener('wrapper:onMessage', ...)}
  \item \code{CapacitorBrowserView.addListener('wrapper:onMenuClick', ...)}
  \item Interfaces
\end{itemize}

% -------------------- %

\paragraph{CapacitorBrowserView.send('wrapper:updateColor', ...)}

\begin{minted}{typescript}
  send: (
    eventName: 'wrapper:updateColor',
    color: string
  ) => void
\end{minted}

Aktualisiert die Hauptfarbe der Benutzeroberfläche.

\begin{arguments}
  \item \code{color}: Die neue Hauptfarbe im \code*{\#RRGGBB}-Format.
\end{arguments}

% -------------------- %

\paragraph{CapacitorBrowserView.send('wrapper:updateMenuColor', ...)}

\begin{minted}{typescript}
  send: (
    eventName: 'wrapper:updateMenuColor',
    color: string
  ) => void
\end{minted}

Ändert die Hintergrundfarbe der Menüleiste.

\begin{arguments}
  \item \code{color}: Die neue Hintergrundfarbe im \code*{\#RRGGBB}-Format.
\end{arguments}

% -------------------- %

\newpage

\paragraph{CapacitorBrowserView.send('wrapper:enableMenu', ...)}

\begin{minted}{typescript}
  send: (
    eventName: 'wrapper:enableMenu',
    enable: boolean
  ) => void
\end{minted}

Aktiviert oder deaktiviert die Menüleiste.

\begin{arguments}
  \item \code{enable}: \code[javascript]{true}, um die Menüleiste zu aktivieren. \code[javascript]{false}, um sie zu deaktivieren.
\end{arguments}

% -------------------- %

\paragraph{CapacitorBrowserView.send('wrapper:addMenuItem', ...)}

\begin{minted}{typescript}
  send: (
    eventName: 'wrapper:addMenuItem',
    item: MenuItem
  ) => void
\end{minted}

Fügt ein \code*{MenuItem} zur Menüleiste hinzu.

\begin{arguments}
  \item \code{item}: Das Menüelement, das hinzugefügt werden soll.
\end{arguments}

% -------------------- %

\paragraph{CapacitorBrowserView.send('wrapper:removeMenuItem', ...)}

\begin{minted}{typescript}
  send: (
    eventName: 'wrapper:removeMenuItem',
    itemId: string
  ) => void
\end{minted}

Entfernt ein Menüelement mit der eindeutigen ID \code{itemId} aus der Menüleiste.

\begin{arguments}
  \item \code{itemId}: Die ID des Menüelements, das entfernt werden soll.
\end{arguments}

% -------------------- %

\paragraph{CapacitorBrowserView.send('wrapper:openMessageScreen', ...)}

\begin{minted}{typescript}
  send: (
    eventName: 'wrapper:openMessageScreen',
    html: string
  ) => void
\end{minted}

Öffnet oder aktualisiert das Meldungsfenster.
Es kann mit \ac{html} angepasst werden und wird automatisch mit Styles des PicoCSS Frameworks versehen.

\begin{arguments}
  \item \code{html}: Der \ac{html}-Code für das Meldungsfenster.
\end{arguments}

% -------------------- %

\newpage

\paragraph{CapacitorBrowserView.send('wrapper:closeMessageScreen')}

\begin{minted}{typescript}
  send: (
    eventName: 'wrapper:closeMessageScreen'
  ) => void
\end{minted}

Schließt das Meldungsfenster.

% -------------------- %

\paragraph{CapacitorBrowserView.send('wrapper:sendMessage', ...)}

\begin{minted}{typescript}
  send: (
    eventName: 'wrapper:sendMessage',
    ...args: any[]
  ) => void
\end{minted}

Sendet eine Nachricht an das Node.js Projekt zusammen mit Argumenten.
Die Argumente werden mit \ac{json} serialisiert.

\begin{arguments}
  \item \code{args}: Eine Liste der Argumente, die gesendet werden sollen.
\end{arguments}

% -------------------- %

\paragraph{CapacitorBrowserView.addListener('wrapper:onMessage', ...)}

\begin{minted}{typescript}
  addListener: (
    eventName: 'wrapper:onMessage',
    listener: (...args: any[]) => void
  ) => void
\end{minted}

Ruft \code{listener(args...)} auf, wenn eine neue Nachricht von dem Node.js Projekt eintrifft.

\begin{arguments}
  \item \code{args}: Eine Liste der Argumente, die empfangenen wurden.
\end{arguments}

% -------------------- %

\paragraph{CapacitorBrowserView.addListener('wrapper:onMenuClick', ...)}

\begin{minted}{typescript}
  addListener: (
    eventName: 'wrapper:onMenuClick',
    listener: (itemId: string) => void
  ) => void
\end{minted}

Ruft \code{listener(data)} auf, wenn auf eine Schaltfläche in der Menüleiste geklickt wurde.

\begin{arguments}
  \item \code{itemId}: Die ID des Menüelements, das angeklickt wurde.
\end{arguments}

% -------------------- %

\newpage

\paragraph{Interfaces}

% -------------------- %

\subparagraph{MenuItem}

Ein Interface, das Informationen über ein Menüelement enthält.

\begin{interfacedesc}{WebApp-Wrapper / MenuItem (Webseite API)}
  \code{id}   & \code[typescript]{string} & Eine eindeutige ID für das Menüelement. \\ \hline
  \code{type} & \code[typescript]{string} & Der Anzeigetyp des Menüelements. Die folgenden Typen sind verfügbar: \\
              &                           & \textbf{\code{text}}: Ein einfaches Textelement. \\
              &                           & \textbf{\code{separator}}: Ein Trennstrich. \\
              &                           & \textbf{\code{button}}: Eine Schaltfläche mit Text. \\
              &                           & \textbf{\code{button_primary}}: Eine Schaltfläche mit Text und hervorgehobener Farbe. \\ \hline
  \code{text} & \code[typescript]{string} & Der Text, der in der Menüleiste für dieses Element angezeigt wird. \\ \hline
\end{interfacedesc}
