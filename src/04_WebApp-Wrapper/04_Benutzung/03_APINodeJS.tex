\subsubsection{API - Node.js Projekt}

Der WebApp-Wrapper bietet im Node.js Projekt eine \ac{api} an, um Konfigurationen zu ändern, die Webseite zu steuern und Nachrichten zwischen dem Node.js Projekt und der Webseite auszutauschen.

Die \acs{api}-Methoden sind über das Modul \code{bridge} verfügbar.
Es verfügt über die folgenden Methoden:

\begin{itemize}
  \setlength\itemsep{-0.8em}
  \item \code{channel.send('wrapper:updateColor', ...)}
  \item \code{channel.send('wrapper:updateMenuColor', ...)}
  \item \code{channel.send('wrapper:enableMenu', ...)}
  \item \code{channel.send('wrapper:addMenuItem', ...)}
  \item \code{channel.send('wrapper:removeMenuItem', ...)}
  \item \code{channel.send('wrapper:openMessageScreen', ...)}
  \item \code{channel.send('wrapper:closeMessageScreen')}
  \item \code{channel.send('wrapper:sendMessage', ...)}
  \item \code{channel.addListener('wrapper:onMessage', ...)}
  \item \code{channel.addListener('wrapper:onMenuClick', ...)}
\end{itemize}

\begin{itemize}
  \setlength\itemsep{-0.8em}
  \item \code{channel.send('wrapper:loadUrl', ...)}
  \item \code{channel.send('wrapper:reload')}
  \item \code{channel.send('wrapper:clearHistory')}
  \item \code{channel.send('wrapper:goBack')}
  \item \code{channel.send('wrapper:goForward')}
  \item \code{channel.send('wrapper:executeJavaScript', ...)}
  \item \code{channel.addListener('will-navigate', ...)}
  \item \code{channel.addListener('did-start-loading', ...)}
  \item \code{channel.addListener('did-finish-load', ...)}
  \item \code{channel.addListener('did-fail-load', ...)}
  \item \code{channel.addListener('dom-ready', ...)}
\end{itemize}

\begin{itemize}
  \setlength\itemsep{-0.8em}
  \item Interfaces
\end{itemize}

% -------------------- %

\paragraph{channel.send('wrapper:updateColor', ...)}

\begin{minted}{typescript}
  send: (
    eventName: 'wrapper:updateColor',
    color: string
  ) => void
\end{minted}

Aktualisiert die Hauptfarbe der Benutzeroberfläche.

\begin{arguments}
  \item \code{color}: Die neue Hauptfarbe im \code*{\#RRGGBB}-Format.
\end{arguments}

% -------------------- %

\newpage

\paragraph{channel.send('wrapper:updateMenuColor', ...)}

\begin{minted}{typescript}
  send: (
    eventName: 'wrapper:updateMenuColor',
    color: string
  ) => void
\end{minted}

Ändert die Hintergrundfarbe der Menüleiste.

\begin{arguments}
  \item \code{color}: Die neue Hintergrundfarbe im \code*{\#RRGGBB}-Format.
\end{arguments}

% -------------------- %

\paragraph{channel.send('wrapper:enableMenu', ...)}

\begin{minted}{typescript}
  send: (
    eventName: 'wrapper:enableMenu',
    enable: boolean
  ) => void
\end{minted}

Aktiviert oder deaktiviert die Menüleiste.

\begin{arguments}
  \item \code{enable}: \code[javascript]{true}, um die Menüleiste zu aktivieren. \code[javascript]{false}, um sie zu deaktivieren.
\end{arguments}

% -------------------- %

\paragraph{channel.send('wrapper:addMenuItem', ...)}

\begin{minted}{typescript}
  send: (
    eventName: 'wrapper:addMenuItem',
    item: MenuItem
  ) => void
\end{minted}

Fügt ein \code*{MenuItem} zur Menüleiste hinzu.

\begin{arguments}
  \item \code{item}: Das Menüelement, das hinzugefügt werden soll.
\end{arguments}

% -------------------- %

\paragraph{channel.send('wrapper:removeMenuItem', ...)}

\begin{minted}{typescript}
  send: (
    eventName: 'wrapper:removeMenuItem',
    itemId: string
  ) => void
\end{minted}

Entfernt ein Menüelement mit der eindeutigen ID \code{itemId} aus der Menüleiste.

\begin{arguments}
  \item \code{itemId}: Die ID des Menüelements, das entfernt werden soll.
\end{arguments}

% -------------------- %

\newpage

\paragraph{channel.send('wrapper:openMessageScreen', ...)}

\begin{minted}{typescript}
  send: (
    eventName: 'wrapper:openMessageScreen',
    html: string
  ) => void
\end{minted}

Öffnet oder aktualisiert das Meldungsfenster.
Es kann mit \ac{html} angepasst werden und wird automatisch mit Stilen des PicoCSS Frameworks versehen.

\begin{arguments}
  \item \code{html}: Der \ac{html}-Code für das Meldungsfenster.
\end{arguments}

% -------------------- %

\paragraph{channel.send('wrapper:closeMessageScreen')}

\begin{minted}{typescript}
  send: (
    eventName: 'wrapper:closeMessageScreen'
  ) => void
\end{minted}

Schließt das Meldungsfenster.

% -------------------- %

\paragraph{channel.send('wrapper:sendMessage', ...)}

\begin{minted}{typescript}
  send: (
    eventName: 'wrapper:updateColor',
    ...args: any[]
  ) => void
\end{minted}

Sendet eine Nachricht an die Webseite zusammen mit Argumenten.
Die Argumente werden mit \ac{json} serialisiert.

\begin{arguments}
  \item \code{args}: Eine Liste der Argumente, die gesendet werden sollen.
\end{arguments}

% -------------------- %

\paragraph{channel.addListener('wrapper:onMessage', ...)}

\begin{minted}{typescript}
  addListener: (
    eventName: 'wrapper:onMessage',
    listener: (...args: any[]) => void
  ) => void
\end{minted}

Ruft \code{listener(args...)} auf, wenn eine neue Nachricht von der Webseite eintrifft.

\begin{arguments}
  \item \code{args}: Eine Liste der Argumente, die empfangenen wurden.
\end{arguments}

% -------------------- %

\newpage

\paragraph{channel.addListener('wrapper:onMenuClick', ...)}

\begin{minted}{typescript}
  addListener: (
    eventName: 'wrapper:onMenuClick',
    listener: (itemId: string) => void
  ) => void
\end{minted}

Ruft \code{listener(data)} auf, wenn auf eine Schaltfläche in der Menüleiste geklickt wurde.

\begin{arguments}
  \item \code{itemId}: Die ID des Menüelements, das angeklickt wurde.
\end{arguments}

% -------------------- %


%! The following APIs are forwarded from the Capacitor-BrowserView plugin
% -------------------- %

\paragraph{channel.send('wrapper:loadUrl', ...)}

\begin{minted}{typescript}
  send(
    eventName: 'wrapper:loadUrl',
    url: string
  ) => void
\end{minted}

Lädt die angegebene \ac{url} in die Ansicht.
Die \ac{url} muss einen Protokollpräfix enthalten, z.B.\ den \code{https://}.

\begin{arguments}
  \item \code{url}: Die \ac{url} einer Webseite.
\end{arguments}

% -------------------- %

\paragraph{channel.send('wrapper:reload')}

\begin{minted}{typescript}
  send(
    eventName: 'wrapper:reload'
  ) => void
\end{minted}

Lädt die aktuelle Webseite neu.

% -------------------- %

\paragraph{channel.send('wrapper:clearHistory')}

\begin{minted}{typescript}
  send(
    eventName: 'wrapper:clearHistory'
  ) => void
\end{minted}

Löscht die interne Liste der Rück- und Vorwärtsnavigation.

% -------------------- %

\paragraph{channel.send('wrapper:goBack')}

\begin{minted}{typescript}
  send(
    eventName: 'wrapper:goBack'
  ) => void
\end{minted}

Bringt den Browser dazu, eine Webseite zurückzugehen.

% -------------------- %

\newpage

\paragraph{channel.send('wrapper:goForward')}

\begin{minted}{typescript}
  send(
    eventName: 'wrapper:goForward'
  ) => void
\end{minted}

Bringt den Browser dazu, eine Webseite weiterzuleiten.

% -------------------- %

\paragraph{channel.send('wrapper:executeJavaScript', ...)}

\begin{minted}{typescript}
  send(
    eventName: 'wrapper:executeJavaScript',
    code: string
  ) => void
\end{minted}

Führt JavaScript asynchron im Kontext der aktuell angezeigten Webseite aus.

\begin{arguments}
  \item \code{code}: Der JavaScript-Code, der ausgeführt werden soll.
\end{arguments}

% -------------------- %

\paragraph{channel.addListener('will-navigate', ...)}

\begin{minted}{typescript}
  addListener(
    eventName: 'will-navigate',
    listenerFunc: (
      url: string,
      isExternal: boolean
    ) => void
  ) => void
\end{minted}

Ruft \code{listenerFunc(data)} auf, wenn ein Benutzer oder die Webseite eine Navigation starten will.
Dies kann geschehen, wenn das Objekt \code[javascript]{window.location} geändert wird oder ein Benutzer auf einen Link auf der Webseite klickt.

Dieses Ereignis wird nicht ausgelöst, wenn die Navigation programmgesteuert mit \acsp{api} gestartet wird, wie \code{wrapper:loadUrl} und \code{wrapper:goBack}, oder für POST"=Anfragen.

Es wird auch nicht für seiteninterne Navigationen ausgelöst, wie das Anklicken von Ankerlinks oder die Aktualisierung des \code[javascript]{window.location.hash}.

\begin{arguments}
  \item \code{url}: Die \ac{url} einer Webseite.
  \item \code{isExternal}: Ob die \ac{url} im externen Browser oder in der BrowserView geöffnet wird.
\end{arguments}

% -------------------- %

\newpage

\paragraph{channel.addListener('did-start-loading', ...)}

\begin{minted}{typescript}
  addListener(
    eventName: 'did-start-loading',
    listenerFunc: () => void
  ) => void
\end{minted}

Ruft \code{listenerFunc()} auf, wenn die Webseite mit dem Laden begonnen hat.

% -------------------- %

\paragraph{channel.addListener('did-finish-load', ...)}

\begin{minted}{typescript}
  addListener(
    eventName: 'did-finish-load',
    listenerFunc: () => void
  ) => void
\end{minted}

Ruft \code{listenerFunc()} auf, wenn das Laden der Webseite abgeschlossen ist.
Dies garantiert nicht, dass der nächste gezeichnete Frame den Zustand des \ac{dom} zu diesem Zeitpunkt wiedergibt.

% -------------------- %

\paragraph{channel.addListener('did-fail-load', ...)}

\begin{minted}{typescript}
  addListener(
    eventName: 'did-fail-load',
    listenerFunc: (error: WebResourceError) => void
  ) => void
\end{minted}

Ruft \code{listenerFunc(data)} auf, wenn die Webseite nicht geladen werden kann.
Diese Fehler weisen in der Regel darauf hin, dass keine Verbindung zum Server hergestellt werden kann.

\begin{arguments}
  \item \code{error}: Informationen über den aufgetretenen Fehler.
\end{arguments}

% -------------------- %

\paragraph{channel.addListener('dom-ready', ...)}

\begin{minted}{typescript}
  addListener(
    eventName: 'dom-ready',
    listenerFunc: () => void
  ) => void
\end{minted}

Ruft \code{listenerFunc()} auf, wenn das Dokument im Frame der obersten Ebene geladen wurde.

% -------------------- %

\newpage

\paragraph{Interfaces}

% -------------------- %

\subparagraph{MenuItem}

Ein Interface, das Informationen über einen Menüelement enthält.

\begin{interfacedesc}{WebApp-Wrapper / MenuItem (Node.js Projekt API)}
  \code{id}   & \code[typescript]{string} & Eine eindeutige ID für das Menüelement. \\ \hline
  \code{type} & \code[typescript]{string} & Der Anzeigetyp des Menüelements. Die folgenden Typen sind verfügbar: \\
              &                           & \textbf{\code{text}}: Ein einfaches Textelement. \\
              &                           & \textbf{\code{separator}}: Ein Trennstrich. \\
              &                           & \textbf{\code{button}}: Eine Schaltfläche mit Text. \\
              &                           & \textbf{\code{button_primary}}: Eine Schaltfläche mit Text und hervorgehobener Farbe. \\ \hline
  \code{text} & \code[typescript]{string} & Der Text, der in der Menüleiste für dieses Element angezeigt wird. \\ \hline
\end{interfacedesc}

\subparagraph{WebResourceError}

Ein Interface, das Informationen über den Fehler enthält, der beim Laden von Webressourcen aufgetreten ist.

\begin{interfacedesc}{WebApp-Wrapper / WebResourceError}
  errorCode         & \code[typescript]{string} & Der Fehlercode des Fehlers. \\ \hline
  errorDescription  & \code[typescript]{string} & Eine Zeichenfolge, die den Fehler beschreibt. Beschreibungen sind lokalisiert und können daher verwendet werden, um dem Benutzer das Problem mitzuteilen. \\ \hline
  validatedURL      & \code[typescript]{string} & Die \ac{url}, die nicht geladen werden konnte. \\ \hline
\end{interfacedesc}
