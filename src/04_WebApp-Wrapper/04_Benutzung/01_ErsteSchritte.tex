\subsubsection{Erste Schritte}

\paragraph{Projektdateien herunterladen}

Um mit der Erstellung einer neuen Anwendung mithilfe des WebApp-Wrappers zu beginnen, müssen zunächst die Projektdateien heruntergeladen und die Abhängigkeiten installiert werden.

Dieser Schritt ist mit den folgenden Befehlen einfach zu bewerkstelligen:

\begin{minted}[breaklines,breakafter=/]{bash}
# Lädt alle erforderlichen Dateien in den Ordner "AppProjectDir" herunter
git clone https://github.com/hampoelz/WebApp-Wrapper.git AppProjectDir

# Wechselt das Arbeitsverzeichnis in den Ordner "AppProjectDir"
cd AppProjectDir

# Installiert alle Abhängigkeiten des WebApp-Wrappers
npm install
\end{minted}

\paragraph{Konfigurationen anpassen}

Bevor die neue Anwendung gestartet werden kann, müssen zunächst die erforderlichen Konfigurationen in der Datei \code{capacitor.config.json} angepasst werden. Die folgenden Konfigurationen sind erforderlich:

\begin{itemize}
  \setlength\itemsep{-0.5em}
  \item \textbf{\code{appId}}: Eine eindeutige Anwendungs-ID
  \item \textbf{\code{appName}}: Der Name der Anwendung
  \item \textbf{\code{appUrl}}: Die \ac{url} der Webseite, die in der Anwendung angezeigt wird.
\end{itemize}

Weitere Konfigurationen können im Kapitel \nameref{sec:WebApp-Wrapper:Konfiguration} gefunden werden.

\newpage

\paragraph{Plattformen hinzufügen}

Nachdem die Konfiguration abgeschlossen ist, können Plattformen hinzugefügt werden, auf denen die Anwendung ausgeführt werden soll.

Die folgenden Befehle können verwendet werden, um Plattformen hinzuzufügen:

\begin{minted}{bash}
# Erzeugt die Projektdateien für die Plattformen
npm run build

# Fügt der Anwendung Unterstützung für eine Plattform hinzu
npx cap add <Plattform>

# Synchronisiert die Plattform mit den Projektdateien
npx cap sync <Plattform>
\end{minted}

Die verfügbaren Plattformen sind \code{android}, \code{ios} und \code{electron}.
Die Plattform \enquote{Electron} bezieht sich auf Desktop"=Plattformen, also Windows, Linux und macOS.

Wenn die Konfiguration später geändert wird, müssen die folgenden Befehle erneut ausgeführt werden:

\begin{minted}{bash}
# Aktualisiert die Projektdateien mit den neuen Konfigurationen
npm run build

# Synchronisiert die angegebenen Plattform mit den Projektdateien
npx cap sync <Plattform>
\end{minted}

\paragraph{Node.js Projekt hinzufügen}

Um zusätzliche Funktionen in die Anwendung zu integrieren, kann ein Node.js Projekt in das Verzeichnis \code{nodejs} hinzugefügt werden.
Dieses Verzeichnis ist bereits vorkonfiguriert, sodass ein Node.js Projekt einfach integrieren werden kann.

Nach dem Hinzufügen oder Ändern des Node.js Projekts müssen die folgenden Befehle ausgeführt werden, um die Projektdateien zu aktualisieren:

\begin{minted}{bash}
# Aktualisiert die Projektdateien mit dem Node.js Projekt
npm run build

# Synchronisiert die Plattform mit den Projektdateien
npx cap sync <Plattform>
\end{minted}

\newpage

\paragraph{Anwendung starten}

Um die Anwendung schließlich zu starten, kann der folgende Befehl verwendet werden.

\begin{minted}{bash}
# Öffnet das Projekt für die angegebene Plattform
npx cap open <Plattform>
\end{minted}

Dieser Befehl öffnet das Projekt für die angegebene Plattform in der entsprechenden Entwicklungsumgebung. 
Bei Android und iOS wird Android Studio bzw.\ Xcode geöffnet, wenn diese auf dem System installiert sind.
In der Entwicklungsumgebung können weitere Aktionen durchgeführt werden, wie das Testen im Emulator, das Debuggen oder das Kompilieren.

Alternativ kann das Projekt für eine Plattform auch manuell geöffnet werden. Die entsprechenden Plattform"=Projektdateien befinden sich in den Verzeichnissen \code{android} und \code{ios}.

Bei Electron wird die Anwendung direkt gestartet. Weitere Aktionen können dabei im Plattform"=Projektordner \code{electron} durchgeführt werden.

