\subsection{Verwendete Bibliotheken}

In diesem Abschnitt werden die Bibliotheken vorgestellt, die bei der Entwicklung der Benutzeroberfläche des WebApp-Wrappers verwendet wurden.

\subsubsection{PicoCSS}

PicoCSS ist ein minimales und responsives Designsystem, das in reinem \ac{css} implementiert ist.
Mit diesem Framework können Webentwickler schnell und einfach responsive Webseiten erstellen.
Das Framework verwendet ausschließlich \acs{html}"=Elemente und -Attribute, um die Gestaltung von Webseiten zu ermöglichen.
Dadurch wird der Code sauberer und leichter zu pflegen.
\cite{pico}

\subsubsection{Material Symbols}

Material Symbols umfasst eine Reihe von Symbolen, die von Google entwickelt wurden.
Die Symbole haben einen einfachen und minimalistischen Stil, sind in verschiedenen Größen erhältlich, einfach zu verwenden und zu verstehen.
\cite{symbols}

\subsubsection{Noto Sans}

Noto Sans ist eine von Google entwickelte Schriftart, die von Millionen von Menschen weltweit verwendet wird.
Die Schrift ist klar und prägnant und in verschiedenen Stärken und -Breiten erhältlich.
Die Schrift unterstützt mehrere Schriftsysteme, darunter Latein, Kyrillisch und Griechisch.
Latein ist das gängigste Schriftsystem der Welt.
Kyrillisch ist ein Alphabet, das für verschiedene Sprachen in ganz Eurasien verwendet wird.
Griechische Symbole werden häufig in wissenschaftlichen Schriften verwendet.
\cite{noto}
