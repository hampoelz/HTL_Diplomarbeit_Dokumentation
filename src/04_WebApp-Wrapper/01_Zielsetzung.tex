\subsection{Zielsetzung}

Das Hauptziel des WebApp-Wrappers ist es, Webentwicklern zu ermöglichen, eine Webseite mit geringem Aufwand in eine Anwendung zu verpacken, die für alle gängigen Desktop- und Mobile"=Plattformen (Windows, Linux, macOS, Android und iOS) kompiliert werden kann.

Um den Funktionsumfang der Anwendung zu erweitern, soll es auch möglich sein, ein Node.js Projekt zu integrieren.

Der WebApp-Wrapper soll eine einfache Benutzeroberfläche enthalten, um Webentwicklern das Erstellen einer intuitiven Benutzererfahrung in der Anwendung zu erleichtern.
Die Oberfläche soll folgende Funktionen enthalten:

\begin{itemize}
  \setlength\itemsep{-0.5em}
  \item \textbf{Ladebildschirm:}
  Informiert den Endbenutzer über das Laden der Webseite.
  \item \textbf{Fehlerbildschirm:}
  Informiert den Endbenutzer über mögliche Fehler, wenn etwa Serverprobleme vorliegen oder die Internetverbindung fehlt.
  \item \textbf{Meldungsfenster:}
  Zeigt Nachrichten an, um den Endbenutzer um eine Eingabe zu bitten oder ihm Informationen zu liefern.
  \item \textbf{Menüleiste:}
  Bietet die Möglichkeit zusätzliche Schaltflächen anzuzeigen, die nicht in der Webseite selbst enthalten sind.
\end{itemize}

Darüber hinaus soll eine \ac{api} bereitgestellt werden, über die die Webseite und das Node.js Projekt miteinander kommunizieren können.
Um die Anpassung der Anwendung an die individuellen Bedürfnisse zu ermöglichen, soll die \ac{api} auf der Webseite sowie im Node.js Projekt die folgenden Funktionen bieten:

\begin{itemize}
  \setlength\itemsep{-0.8em}
  \item Steuern der geladenen Webseite
  \item Ändern der Hauptfarbe von der Benutzeroberfläche
  \item Anpassen, Anzeigen und Ausblenden des Meldungsfensters
  \item Nachträgliches Ändern, Anzeigen und Ausblenden der Menüleiste
\end{itemize}
