\addurlfn{electron}{Electron}{https://electronjs.org/}
\addurlfn{nodejs}{Node.js}{https://nodejs.org/}
\addurlfn{capacitor}{Capacitor}{https://capacitorjs.com/}
\addurlfn{browserview}{BrowserView-API}{https://electronjs.org/docs/latest/api/browser-view}

\unnumberedSection{Kurzbeschreibung}

Durch das Framework \fn{electron} werden Desktop"=Anwendungen mit Webtechnologien \textit{(Frontend)} und \fn{nodejs} \textit{(Backend)} entwickelt.
Um Mobile"=Anwendungen mit Webtechnologien zu entwickeln, steht das \fn{capacitor}-Framework zur Verfügung.
Allerdings fehlen dort Funktionen wie ein \fn{nodejs} Backend oder eine \fn{browserview} um zusätzliche Webinhalte einzubetten.

Im Rahmen dieser Diplomarbeit wurden die Erweiterungen \textsc{Capacitor-NodeJS} und \textsc{Capacitor-BrowserView} entwickelt,
um ein \fn{nodejs} Backend in \fn{capacitor} zu integrieren und die \fn{browserview} von \fn{electron} zu portiert.

Mit den Frameworks \fn{capacitor} und \fn{electron}, und den beiden Erweiterungen, wurde ein plattformübergreifender WebApp-Wrapper erstellt,
welcher über eine Benutzeroberfläche für den Fall von Internetproblemen oder anderen Ereignissen verfügt.
Darüber hinaus werden mehrere APIs zur Steuerung des WebApp-Wrappers und der Web-App bereitgestellt.

Entwickler können damit ohne großen Aufwand eine plattformübergreifende Anwendungen mit zusätzlichem Backend für ihre Website/Web-App erstellen. 

\unnumberedSection{Abstract}

The \fn{electron} framework is used to develop desktop applications using web technologies \textit{(frontend)} and \fn{nodejs} \textit{(backend)}.
To develop mobile apps using web technologies, the \fn{capacitor} framework is used.
However, it lacks features like a \fn{nodejs} backend or a \fn{browserview} to embed additional web content.

As part of this thesis, the extensions \textsc{Capacitor-NodeJS} and \textsc{Capacitor-BrowserView} were developed
to integrate a \fn{nodejs} backend into \fn{capacitor} and to port the \fn{browserview} from \fn{electron}.

Using the \fn{capacitor} and \fn{electron} frameworks, and the two extensions, a cross-platform WebApp-Wrapper was developed,
which features a user interface in case of internet problems or other events.
In addition, several APIs are provided to control the WebApp-Wrapper and the web app.

Developers can use it to create a cross-platform app with an additional backend for their website/web-app without much effort.

\printfn
