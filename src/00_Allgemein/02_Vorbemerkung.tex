\unnumberedSection{Projektteam \& Einteilung}

\begin{description}
    \item[Teil 1 - Capacitor-NodeJS] \hfill\\
    Programmiersprachen: TypeScript, JavaScript, C/C++, Java, Swift\\
    Diplomand: Rene Hampölz\\
    GitHub: \href{https://github.com/hampoelz/Capacitor-NodeJS}{https://github.com/hampoelz/Capacitor-NodeJS}
    \item[Teil 2 - Capacitor-BrowserView] \hfill\\
    Programmiersprachen: TypeScript, JavaScript, Java, Swift\\
    Diplomand: Rene Hampölz\\
    GitHub: \href{https://github.com/hampoelz/Capacitor-BrowserView}{https://github.com/hampoelz/Capacitor-BrowserView}
    \item[Teil 3 - WebApp-Wrapper] \hfill\\
    Programmiersprachen: HTML, CSS, JavaScript, TypeScript\\
    Diplomand: Noah Quinz\\
    GitHub: \href{https://github.com/hampoelz/WebApp-Wrapper}{https://github.com/hampoelz/WebApp-Wrapper}
\end{description}

\begin{note}
    Für die Dokumentation dieses Projekts wurde die HTL Weiz LaTeX Vorlage für Diplomarbeiten verwendet.~\cite{template:latex} Diese Vorlage wurde vom Diplomand Rene Hampölz zeitgleich mit der vorliegenden Arbeit entwickelt.
    Sie ist speziell für Diplomarbeiten an der HTL Weiz konzipiert und soll sowohl Neulingen den Einstieg in LaTeX erleichtern als auch erfahrenen LaTeX-Nutzern die Arbeit erleichtern.
\end{note}

% ------ Änderungsverlauf ------
\printChangelog
