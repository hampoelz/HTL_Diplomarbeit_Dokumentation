\subsubsection{Electron}

Electron ist ein plattformübergreifendes Framework, das es Webentwicklern ermöglicht, Desktop"=Anwendungen für Windows, macOS und Linux zu erstellen.
Es basiert auf Chromium, einem Webbrowser, und Node.js, einer Plattform für die Ausführung von JavaScript Code außerhalb eines Webbrowsers.
\cite{electron:docs}

Anwendungen werden mit JavaScript, \ac{html}, \ac{css} und Node.js erstellt.
Der Zugriff auf Funktionen des Betriebssystems erfolgt über die von Electron bereitgestellte \acp{api}.
\cite{electron:docs}

Electron wird häufig verwendet, um bestehende Webanwendungen in Desktop"=Anwendungen mit einem ähnlichen Benutzererlebnis umzuwandeln.
Dadurch wird Electron von einer Reihe beliebter Anwendungen genutzt, darunter Discord, Microsoft Teams, Notion und VS Code.
\cite{electron}
