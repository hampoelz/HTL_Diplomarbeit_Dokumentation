\subsection{Benutzung}

In diesem Kapitel wird erklärt, wie das Plugin verwendet wird.

Um das Plugin für eine globale Zielgruppe zugänglich zu machen, wurde die nachfolgende Dokumentation in englischer Sprache verfasst.

\vspace{2em}

\textbf{Table of contents}

\begin{itemize}
  \setlength\itemsep{-1em}
  \item \nameref{sec:Capacitor-NodeJS:Install}
  \item \nameref{sec:Capacitor-NodeJS:GettingStarted}
  \vspace{\itemsep}
  \begin{itemize}
    \setlength\itemsep{-1em}
    \item \nameref{sec:Capacitor-NodeJS:Basics}
    \item \nameref{sec:Capacitor-NodeJS:MinimalExample}
    \item \nameref{sec:Capacitor-NodeJS:InterprocessCommunication}
  \end{itemize}
  \item \nameref{sec:Capacitor-NodeJS:ComplexProjects}
  \vspace{\itemsep}
  \begin{itemize}
    \setlength\itemsep{-1em}
    \item \nameref{sec:Capacitor-NodeJS:CustomStartingPoint}
    \item \nameref{sec:Capacitor-NodeJS:InstallModules}
    \item \nameref{sec:Capacitor-NodeJS:ImproveLoadingTimes}
    \item \nameref{sec:Capacitor-NodeJS:ManualRuntimeStart}
  \end{itemize}
  \item \nameref{sec:Capacitor-NodeJS:MobileAPIDifferences}
  \item \nameref{sec:Capacitor-NodeJS:Configuration}
  \item \nameref{sec:Capacitor-NodeJS:API_BridgeModule}
  \item \nameref{sec:Capacitor-NodeJS:API_CapacitorLayer}
\end{itemize}

\vspace{1em}

The following documentation refers to the pre-release version \code{1.0.0-beta.4} of the plugin.
The latest version as well as the related documentation can be found on the GitHub project page\footnote{\url{https://github.com/hampoelz/Capacitor-NodeJS}}.

\clearpage

\subsubsection{Install}
\label{sec:Capacitor-NodeJS:Install}

Capacitor v5 or newer is required. This project isn't compatible with lower versions of Capacitor.

\begin{minted}[breaklines,breakafter=/]{bash}
npm install https://github.com/hampoelz/capacitor-nodejs/releases/download/v1.0.0-beta.5/capacitor-nodejs.tgz
npx cap sync
\end{minted}

\begin{note}[Note]
  For now Android 32-bit x86 support is disabled since Capacitor-NodeJS \code{v1.0.0-beta.2} as there is currently no support for it in the latest version of the Node.js for Mobile Apps toolkit.
  \cite{nodejs-mobile}
\end{note}

\subsubsection{Getting Started}
\label{sec:Capacitor-NodeJS:GettingStarted}

This guide shows how to add a minimal Node.js project to a Capacitor application and communicate between these processes.

\paragraph{Basics}
\label{sec:Capacitor-NodeJS:Basics}

In the example below the Vite build system is used.
However, any build system can be used as long as the following criteria are met:

\begin{enumerate}
  \item The Node.js project (to be executed by the engine) must be located in a subdirectory named \code{nodejs} \textit{(or the path set via \code{nodeDir})} of the Capacitor \code{webDir}.
  \item The Node.js project must have a starting point, this can either be a script named \code{index.js} or a package.json with a \code{main} field.
\end{enumerate}

\begin{quote}
  For example if the Node.js project needs to be compiled or bundled then this output should be located in the subdirectory of the Capacitor \code{webDir}.
\end{quote}

\newpage

\paragraph{Minimal example}
\label{sec:Capacitor-NodeJS:MinimalExample}

In this example the directory for the app's source files is named \code{src}, the directory for static assets is named \code{static},
the directory for the compiled files is named \code{dist}, and the directory for the Node.js project is named \code{nodejs}.

So the configurations should contain at least the following values.~\cite{vite, capacitor:docs}

\textbf{Vite Configurations:}

\begin{minted}{typescript}
// in vite.config.js or vite.config.ts
{
  root: './src',
  publicDir: '../static',
  build: {
    outDir: '../dist'
  }
}
\end{minted}

\textbf{Capacitor Configurations:}

\begin{minted}{typescript}
// in capacitor.config.json or capacitor.config.ts
{
  "webDir": 'dist',
  "plugins": {
    "CapacitorNodeJS": {
      "nodeDir": "nodejs"
    }
  }
}
\end{minted}

\vspace{1em}

To meet the criteria from above using Vite, just create a new directory called \code{nodejs} inside the \code{static} directory.
And create a new file called \code{index.js} in it as the starting point.

\begin{quote}
  Vite will copy assets from the \code{static} directory to the root of the \code{dist} directory as-is.~\cite{vite}
  So the created \code{nodejs} project directory will be placed in the Capacitor \code{webdir} after build.
\end{quote}

\vspace{1em}

The project structure should now look something like this:

\begin{minted}{diff}
  capacitor-app/
  ├── ...
  ├── dist/                   # Capacitor webdir
  ├── src/                    # app source directory
+ ├── static/                 # static assets
+ │   ├── nodejs/             # Node.js project directory
+ │   │   ├── index.js        # Node.js main script
  ├── capacitor.config.json
  ├── vite.config.ts
  ├── ...
\end{minted}

\vspace{1em}

After building and syncing the project, the main script will be executed by the Node.js runtime when the app is launched.

A guide for a more complex Node.js project can be found in the \nameref{sec:Capacitor-NodeJS:ComplexProjects} section.

\paragraph{Inter-Process Communication}
\label{sec:Capacitor-NodeJS:InterprocessCommunication}

A bridge module to communicate between the Capacitor layer and the Node.js process is built-in.

Use the following code in a Node.js script to wait for messages from the Capacitor layer and send messages back:

\begin{minted}{javascript}
const { channel } = require('bridge');

// Listens to "msg-from-capacitor" from the Capacitor layer.
channel.addListener('msg-from-capacitor', message => {
  console.log('[Node.js] Message from Capacitor: ' + message);
  
  // Sends a message back to the Capacitor layer.
  channel.send("msg-from-nodejs",
    `Replying to the message '${message}'.`,
    "And optionally add more arguments."
  );
});
\end{minted}

\vspace{1em}

Now it is possible to communicate with the Node.js process in the Capacitor app:

\begin{minted}{typescript}
import { NodeJS } from 'capacitor-nodejs';

// Listens to "msg-from-nodejs" from the Node.js process.
NodeJS.addListener('msg-from-nodejs', event => {
  document.body.innerHTML = `
    <p>
      <b>Message from Capacitor</b><br>
      First argument: ${event.args[0]}<br>
      Second argument: ${event.args[1]}
    </p>
  `;
  console.log(event);
});

// Waits for the Node.js process to initialize.
NodeJS.whenReady().then(() => {

  // Sends a message to the Node.js process.
  NodeJS.send({
    eventName: "msg-from-capacitor",
    args: [ "Hello from Capacitor!" ]
  });

});
\end{minted}

A full \ac{api} documentation can be found in the \nameref{sec:Capacitor-NodeJS:API_BridgeModule} section.

\clearpage

\input{src/02_Capacitor-NodeJS/04_Benutzung/03_ComplexProjects.tex}
\clearpage

\subsubsection{Mobile Node.js APIs differences}
\label{sec:Capacitor-NodeJS:MobileAPIDifferences}

\begin{note}[Note]
  This section is based on the documentation of the Node.js for Mobile Apps toolkits.
  \cite{nodejs-mobile:docs}
\end{note}

Not every \ac{api} is supported on mobile devices.
Mobile operating systems do not allow applications to call certain \acsp{api} that are expected to be available on other operating systems.

\paragraph{child\_process module}

Mobile applications are expected to be a single process.
\acsp{api} that create new processes, such as \code{child_process.spawn()} or \code{child_process.fork()} will therefore run into permission issues.

\paragraph{file system (fs) module}

On mobile platforms, the current working directory is the root directory of the file system.
This can lead to unexpected behavior in code that assumes that the current working directory is set to the directory of the Node.js project.

On Android creating hard links (\code{fs.link()} and \code{fs.linkSync()}) is not supported.

\paragraph{internationalization (intl) module}

The internationalization (\code{intl}) module is not available on current nodejs-mobile builds.

\paragraph{os module}

\begin{itemize}
  \setlength\itemsep{-0.5em}
  \item \code{os.cpus()} may return inconsistent/unreliable results, since different OS versions will have different permissions for accessing CPU information.
  \item \code{os.homedir()} on mobile platforms there is no concept of user home directories.
  \item \code{os.platform()} can also return \enquote{android} or \enquote{ios}, depending on the platform.
\end{itemize}

On Android, the files in the cache (\code{os.tmpdir()}) are kept until the system needs space, so it increases the application's disk space unless the developer manually deletes them.

\newpage

\paragraph{process module}

\begin{itemize}
  \setlength\itemsep{-0.5em}
  \item \code{process.cwd()} is the root directory of the file system, instead of the start directory of the project.
  \item \code{process.exit()} is not allowed by the Apple App Store guildelines.
  \item \code{process.stdin} is not available.
  \item \code{process.platform} can also be \enquote{android} or \enquote{ios}, depending on the platform.
  \item \code{process.versions} includes the \enquote{mobile} key, containing the nodejs-mobile core library version.
\end{itemize}

The following functions are only available on POSIX platforms, so they are unavailable on Android:

\begin{itemize}
  \setlength\itemsep{-0.8em}
  \item \code{process.getegid()}
  \item \code{process.geteuid()}
  \item \code{process.getgid()}
  \item \code{process.getgroups()}
  \item \code{process.getuid()}
  \item \code{process.setegid()}
  \item \code{process.seteuid()}
  \item \code{process.setgid()}
  \item \code{process.setgroups()}
  \item \code{process.setuid()}
\end{itemize}

\clearpage

\subsubsection{Configuration}
\label{sec:Capacitor-NodeJS:Configuration}

These config values are available:

\begin{configuration}{Capacitor-NodeJS / Configuration}
  \code{nodeDir}   & \code[typescript]{string} & Relative path of the integrated Node.js project based on the Capacitor webdir. Defaults to \code[typescript]{"nodejs"}. \\ \hline
  \code{startMode} & \code[typescript]{string} & Startup mode of the Node.js engine. Defaults to \code[typescript]{"auto"}. The following values are accepted: \\
                   &                           & \textbf{\code{auto}}: The Node.js engine starts automatically when the application is launched. \\
                   &                           & \textbf{\code{manual}}: The Node.js engine is started via the \code{NodeJS.start()} method. \\ \hline
\end{configuration}
  
\textbf{Examples}

In \code{capacitor.config.json}:

\begin{minted}{json}
{
  "plugins": {
    "CapacitorNodeJS": {
      "nodeDir": "custom-nodejs",
      "startMode": "manual"
    }
  }
}
\end{minted}

In \code{capacitor.config.ts}:

\begin{minted}{typescript}
/// <reference types="capacitor-nodejs" />

import { CapacitorConfig } from '@capacitor/cli';

const config: CapacitorConfig = {
  plugins: {
    CapacitorNodeJS: {
      nodeDir: "custom-nodejs",
      startMode: "manual",
    },
  },
};

export default config;
\end{minted}

\clearpage

\subsubsection{API - Bridge module}
\label{sec:Capacitor-NodeJS:API_BridgeModule}

The \code{bridge} module is built-in.
It provides an \ac{api} to communicate between the Capacitor layer and the Node.js process, as well as an \ac{api} to get a per-user application data directory on each platform.

TypeScript declarations for this \code{bridge} module can be manually installed as dev-dependency.
If needed, the types-only package can be found under \code{node_modules/capacitor-nodejs/assets/types/bridge} in the root of the Capacitor project.

% -------------------- %

\begin{itemize}
  \setlength\itemsep{-0.8em}
  \item \code{getDataPath()}
  \item \code{channel}
\end{itemize}

% -------------------- %

\paragraph{getDataPath()}

\begin{minted}{typescript}
  getDataPath: () => string
\end{minted}

Returns a path for a per-user application data directory on each platform, where data can be read and written.

% -------------------- %

\paragraph{channel}

The \code{channel} class of the \code{bridge} module is an Event-Emitter.
It provides a few methods to send messages from the Node.js process to the Capacitor layer, and to receive replies from the Capacitor layer.

It has the following method to listen for events and send messages:

\begin{itemize}
  \setlength\itemsep{-0.8em}
  \item \code{send(...)}
  \item \code{on(string, ...)}
  \item \code{once(string, ...)}
  \item \code{addListener(string, ...)}
  \item \code{removeListener(...)}
  \item \code{removeAllListeners(...)}
\end{itemize}

% -------------------- %

\paragraph{channel.send(...)}

\begin{minted}{typescript}
  send: (
    eventName: string,
    ...args: any[]
  ) => void
\end{minted}

Sends a message to the Capacitor layer via \code{eventName}, along with arguments.
Arguments will be serialized with \ac{json}.

\begin{itemize}
  \setlength\itemsep{-0.8em}
  \item \code{eventName}: The name of the event being send to.
  \item \code{args}: The Array of arguments to send.
\end{itemize}

% -------------------- %

\paragraph{channel.on(string, ...)}

\begin{minted}{typescript}
  on: (
    eventName: string,
    listener: (...args: any[]) => void
  ) => void
\end{minted}

Listens to \code{eventName} and calls \code{listener(args...)} when a new message arrives from the Capacitor layer.

\begin{itemize}
  \setlength\itemsep{-0.8em}
  \item \code{args}: The received array of arguments.
\end{itemize}

% -------------------- %

\paragraph{channel.once(string, ...)}

\begin{minted}{typescript}
  once: (
    eventName: string,
    listener: (...args: any[]) => void
  ) => void
\end{minted}

Listens one time to \code{eventName} and calls \code{listener(args...)} when a new message arrives from the Capacitor layer, after which it is removed.

\begin{itemize}
  \setlength\itemsep{-0.8em}
  \item \code{args}: The received array of arguments.
\end{itemize}

% -------------------- %

\paragraph{channel.addListener(string, ...)}

\begin{minted}{typescript}
  addListener: (
    eventName: string,
    listener: (...args: any[]) => void
  ) => void
\end{minted}

Alias for \code{channel.on(string, ...)}.

% -------------------- %

\paragraph{channel.removeListener(...)}

\begin{minted}{typescript}
  removeListener: (
    eventName: string,
    listener: (...args: any[]) => void
  ) => void
\end{minted}

Removes the specified \code{listener} from the listener array for the specified \code{eventName}.

% -------------------- %

\paragraph{channel.removeAllListeners(...)}

\begin{minted}{typescript}
  removeAllListeners: (
    eventName?: string
  ) => void
\end{minted}

Removes all listeners, or those of the specified \code{eventName}.

\begin{itemize}
  \setlength\itemsep{-0.8em}
  \item \code{eventName}: The name of the event all listeners will be removed from.
\end{itemize}

% -------------------- %

\clearpage

\subsubsection{API - Capacitor layer}
\label{sec:Capacitor-NodeJS:API_CapacitorLayer}

The \code*{NodeJS} module is the \ac{api} used in the Capacitor application.
It provides a few methods to send messages from the Node.js layer and wait for them.

It has the following methods:

\begin{itemize}
  \setlength\itemsep{-0.8em}
  \item \code{start(...)}
  \item \code{send(...)}
  \item \code{whenReady()}
  \item \code{addListener(string, ...)}
  \item \code{removeListener(...)}
  \item \code{removeAllListeners(...)}
  \item Interfaces
  \item Type Aliases
\end{itemize}

% -------------------- %

\paragraph{start(...)}

\begin{minted}{typescript}
  start(
    options?: StartOptions
  ) => Promise<void>
\end{minted}

Starts the Node.js engine with properties as set by the \code{options}.

\textbf{Note:} This method is only available if the Node.js engine startup mode was set to \code[typescript]{'manual'} via the plugin configuration.

% -------------------- %

\paragraph{send(...)}

\begin{minted}{typescript}
  send(
    args: ChannelPayloadData
  ) => Promise<void>
\end{minted}

Sends a message to the Node.js process.

% -------------------- %

\paragraph{whenReady()}

\begin{minted}{typescript}
  whenReady() => Promise<void>
\end{minted}

Resolves when the Node.js process is initialized.

% -------------------- %

\newpage

\paragraph{addListener(string, ...)}

\begin{minted}{typescript}
  addListener(
    eventName: string,
    listenerFunc: ChannelListenerCallback
  ) => Promise<PluginListenerHandle> & PluginListenerHandle
\end{minted}

Listens to \code{eventName} and calls \code{listenerFunc(data)} when a new message arrives from the Node.js process.

\textbf{Note:} When using the Electron platform, \code{PluginListenerHandle.remove()} does not work due to limitations.~\cite{capacitor-electron}
Use \code{removeListener(listenerFunc)} instead.

\textbf{Returns:} \code[typescript]{Promise<PluginListenerHandle> & PluginListenerHandle}

% -------------------- %

\paragraph{removeListener(...)}

\begin{minted}{typescript}
  removeListener(
    listenerHandle: PluginListenerHandle
  ) => Promise<void>
\end{minted}

Removes the specified \code{listenerHandle} from the listener array for the event it refers to.

% -------------------- %

\paragraph{removeAllListeners(...)}

\begin{minted}{typescript}
  removeAllListeners(
    eventName?: string
  ) => Promise<void>
\end{minted}

Removes all listeners, or those of the specified \code{eventName}, for this plugin.

% -------------------- %

\newpage

\paragraph{Interfaces}

% -------------------- %

\subparagraph{StartOptions}

An interface containing the options used when starting the Node.js engine manually.

\begin{interfacedesc}{Capacitor-NodeJS / StartOptions}[4.5em][5em]
  \code{nodeDir} & \code[typescript]{string}   & Relative path of the integrated Node.js project based on the Capacitor webdir. Defaults to the \code{nodeDir} field of the global plugin configuration. \\ \hline
  \code{script}  & \code[typescript]{string}   & The primary entry point to the Node.js program. This should be a module relative to the root of the Node.js project folder. Defaults to the \code{main} field in the project's package.json. If the \code{main} field is not set, \code{index.js} in the project's root folder is used. \\ \hline
  \code{args}    & \code[typescript]{string[]} & A list of string arguments. \\ \hline
  \code{env}     & \code[typescript]{NodeEnv}  & Environment key-value pairs. \\ \hline
\end{interfacedesc}

% -------------------- %

\subparagraph{NodeEnv}

An interface that holds environment variables as string key-value pairs.

% -------------------- %

\subparagraph{ChannelPayloadData}

The payload data to send a message to the web page via \code{eventName}, along with arguments.
Arguments will be serialized with \ac{json}.

\begin{interfacedesc}{Capacitor-NodeJS / ChannelPayloadData}
  \code{eventName} & \code[typescript]{string} & The name of the event being send to. \\ \hline
  \code{args}      & \code[typescript]{any[]}  & The array of arguments to send. \\ \hline
\end{interfacedesc}   

% -------------------- %

\subparagraph{ChannelCallbackData}

The callback data object when a message from the Node.js process arrives.

\begin{interfacedesc}{Capacitor-NodeJS / ChannelCallbackData}
  \code{args} & \code[typescript]{any[]} & The received array of arguments. \\ \hline
\end{interfacedesc}

% -------------------- %

\newpage

\paragraph{Type Aliases}

% -------------------- %

\subparagraph{ChannelListenerCallback}

The callback function to be called when listen to messages from the Node.js process.

\code[typescript]{(data: ChannelCallbackData): void}

