\subsubsection{Getting Started}
\label{sec:Capacitor-NodeJS:GettingStarted}

This guide shows how to add a minimal Node.js project to a Capacitor application and communicate between these processes.

\paragraph{Basics}
\label{sec:Capacitor-NodeJS:Basics}

In the example below the Vite build system is used.
However, any build system can be used as long as the following criteria are met:

\begin{enumerate}
  \item The Node.js project (to be executed by the engine) must be located in a subdirectory named \code{nodejs} \textit{(or the path set via \code{nodeDir})} of the Capacitor \code{webDir}.
  \item The Node.js project must have a starting point, this can either be a script named \code{index.js} or a package.json with a \code{main} field.
\end{enumerate}

\begin{quote}
  For example if the Node.js project needs to be compiled or bundled then this output should be located in the subdirectory of the Capacitor \code{webDir}.
\end{quote}

\newpage

\paragraph{Minimal example}
\label{sec:Capacitor-NodeJS:MinimalExample}

In this example the directory for the app's source files is named \code{src}, the directory for static assets is named \code{static},
the directory for the compiled files is named \code{dist}, and the directory for the Node.js project is named \code{nodejs}.

So the configurations should contain at least the following values.~\cite{vite, capacitor:docs}

\textbf{Vite Configurations:}

\begin{minted}{typescript}
// in vite.config.js or vite.config.ts
{
  root: './src',
  publicDir: '../static',
  build: {
    outDir: '../dist'
  }
}
\end{minted}

\textbf{Capacitor Configurations:}

\begin{minted}{typescript}
// in capacitor.config.json or capacitor.config.ts
{
  "webDir": 'dist',
  "plugins": {
    "CapacitorNodeJS": {
      "nodeDir": "nodejs"
    }
  }
}
\end{minted}

\vspace{1em}

To meet the criteria from above using Vite, just create a new directory called \code{nodejs} inside the \code{static} directory.
And create a new file called \code{index.js} in it as the starting point.

\begin{quote}
  Vite will copy assets from the \code{static} directory to the root of the \code{dist} directory as-is.~\cite{vite}
  So the created \code{nodejs} project directory will be placed in the Capacitor \code{webdir} after build.
\end{quote}

\vspace{1em}

The project structure should now look something like this:

\begin{minted}{diff}
  capacitor-app/
  ├── ...
  ├── dist/                   # Capacitor webdir
  ├── src/                    # app source directory
+ ├── static/                 # static assets
+ │   ├── nodejs/             # Node.js project directory
+ │   │   ├── index.js        # Node.js main script
  ├── capacitor.config.json
  ├── vite.config.ts
  ├── ...
\end{minted}

\vspace{1em}

After building and syncing the project, the main script will be executed by the Node.js runtime when the application is launched.

A guide for a more complex Node.js project can be found in the \nameref{sec:Capacitor-NodeJS:ComplexProjects} section.

\paragraph{Interprocess Communication}
\label{sec:Capacitor-NodeJS:InterprocessCommunication}

A bridge module to communicate between the Capacitor layer and the Node.js process is built-in.

Use the following code in a Node.js script to wait for messages from the Capacitor layer and send messages back:

\begin{minted}{javascript}
const { channel } = require('bridge');

// Listens to "msg-from-capacitor" from the Capacitor layer.
channel.addListener('msg-from-capacitor', message => {
  console.log('[Node.js] Message from Capacitor: ' + message);
  
  // Sends a message back to the Capacitor layer.
  channel.send("msg-from-nodejs",
    `Replying to the message '${message}'.`,
    "And optionally add more arguments."
  );
});
\end{minted}

\vspace{1em}

Now it is possible to communicate with the Node.js process in the Capacitor application:

\begin{minted}{typescript}
import { NodeJS } from 'capacitor-nodejs';

// Listens to "msg-from-nodejs" from the Node.js process.
NodeJS.addListener('msg-from-nodejs', event => {
  document.body.innerHTML = `
    <p>
      <b>Message from Capacitor</b><br>
      First argument: ${event.args[0]}<br>
      Second argument: ${event.args[1]}
    </p>
  `;
  console.log(event);
});

// Waits for the Node.js process to initialize.
NodeJS.whenReady().then(() => {

  // Sends a message to the Node.js process.
  NodeJS.send({
    eventName: "msg-from-capacitor",
    args: [ "Hello from Capacitor!" ]
  });

});
\end{minted}

A full \ac{api} documentation can be found in the \nameref{sec:Capacitor-NodeJS:API_BridgeModule} section.
