\subsubsection{Mobile Node.js APIs differences}
\label{sec:Capacitor-NodeJS:MobileAPIDifferences}

\begin{note}[Note]
  This section is based on the documentation of the Node.js for Mobile Apps toolkits.
  \cite{nodejs-mobile:docs}
\end{note}

Not every \ac{api} is supported on mobile devices.
Mobile operating systems do not allow applications to call certain \acsp{api} that are expected to be available on other operating systems.

\paragraph{child\_process module}

Mobile applications are expected to be a single process.
\acsp{api} that create new processes, such as \code{child_process.spawn()} or \code{child_process.fork()} will therefore run into permission issues.

\paragraph{file system (fs) module}

On mobile platforms, the current working directory is the root directory of the file system.
This can lead to unexpected behavior in code that assumes that the current working directory is set to the directory of the Node.js project.

On Android creating hard links (\code{fs.link()} and \code{fs.linkSync()}) is not supported.

\paragraph{internationalization (intl) module}

The internationalization (\code{intl}) module is not available on current nodejs-mobile builds.

\paragraph{os module}

\begin{itemize}
  \setlength\itemsep{-0.5em}
  \item \code{os.cpus()} may return inconsistent/unreliable results, since different OS versions will have different permissions for accessing CPU information.
  \item \code{os.homedir()} on mobile platforms there is no concept of user home directories.
  \item \code{os.platform()} can also return \enquote{android} or \enquote{ios}, depending on the platform.
\end{itemize}

On Android, the files in the cache (\code{os.tmpdir()}) are kept until the system needs space, so it increases the application's disk space unless the developer manually deletes them.

\newpage

\paragraph{process module}

\begin{itemize}
  \setlength\itemsep{-0.5em}
  \item \code{process.cwd()} is the root directory of the file system, instead of the start directory of the project.
  \item \code{process.exit()} is not allowed by the Apple App Store guildelines.
  \item \code{process.stdin} is not available.
  \item \code{process.platform} can also be \enquote{android} or \enquote{ios}, depending on the platform.
  \item \code{process.versions} includes the \enquote{mobile} key, containing the nodejs-mobile core library version.
\end{itemize}

The following functions are only available on POSIX platforms, so they are unavailable on Android:

\begin{itemize}
  \setlength\itemsep{-0.8em}
  \item \code{process.getegid()}
  \item \code{process.geteuid()}
  \item \code{process.getgid()}
  \item \code{process.getgroups()}
  \item \code{process.getuid()}
  \item \code{process.setegid()}
  \item \code{process.seteuid()}
  \item \code{process.setgid()}
  \item \code{process.setgroups()}
  \item \code{process.setuid()}
\end{itemize}
