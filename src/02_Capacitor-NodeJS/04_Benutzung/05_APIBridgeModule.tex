\addurlfn{event-emitter}{Event-Emitter}{https://nodejs.org/api/events.html}

\subsubsection{API - Bridge module}
\label{sec:Capacitor-NodeJS:API_BridgeModule}

The \code{bridge} module is built-in.
It provides an \ac{api} to communicate between the Capacitor layer and the Node.js process, as well as an \ac{api} to get a per-user application data directory on each platform.

TypeScript declarations for this \code{bridge} module can be manually installed as dev-dependency.
If needed, the types-only package can be found under \code{node_modules/capacitor-nodejs/assets/types/bridge} in the root of the Capacitor project.

% -------------------- %

\begin{itemize}
  \setlength\itemsep{-0.8em}
  \item \code{getDataPath()}
  \item \code{channel}
\end{itemize}

% -------------------- %

\paragraph{getDataPath()}

\begin{minted}{typescript}
  getDataPath: () => string
\end{minted}

Returns a path for a per-user application data directory on each platform, where data can be read and written.

% -------------------- %

\paragraph{channel}

The \code{channel} class of the \code{bridge} module is an \fn{event-emitter}.
It provides a few methods to send messages from the Node.js process to the Capacitor layer, and to receive replies from the Capacitor layer.

It has the following method to listen for events and send messages:

\begin{itemize}
  \setlength\itemsep{-0.8em}
  \item \code{send(...)}
  \item \code{on(string, ...)}
  \item \code{once(string, ...)}
  \item \code{addListener(string, ...)}
  \item \code{removeListener(...)}
  \item \code{removeAllListeners(...)}
\end{itemize}

% -------------------- %

\paragraph{channel.send(...)}

\begin{minted}{typescript}
  send: (
    eventName: string,
    ...args: any[]
  ) => void
\end{minted}

Sends a message to the Capacitor layer via \code{eventName}, along with arguments.
Arguments will be serialized with \ac{json}.

\begin{itemize}
  \setlength\itemsep{-0.8em}
  \item \code{eventName}: The name of the event being send to.
  \item \code{args}: The Array of arguments to send.
\end{itemize}

% -------------------- %

\paragraph{channel.on(string, ...)}

\begin{minted}{typescript}
  on: (
    eventName: string,
    listener: (...args: any[]) => void
  ) => void
\end{minted}

Listens to \code{eventName} and calls \code{listener(args...)} when a new message arrives from the Capacitor layer.

\begin{itemize}
  \setlength\itemsep{-0.8em}
  \item \code{args}: The received array of arguments.
\end{itemize}

% -------------------- %

\paragraph{channel.once(string, ...)}

\begin{minted}{typescript}
  once: (
    eventName: string,
    listener: (...args: any[]) => void
  ) => void
\end{minted}

Listens one time to \code{eventName} and calls \code{listener(args...)} when a new message arrives from the Capacitor layer, after which it is removed.

\begin{itemize}
  \setlength\itemsep{-0.8em}
  \item \code{args}: The received array of arguments.
\end{itemize}

% -------------------- %

\paragraph{channel.addListener(string, ...)}

\begin{minted}{typescript}
  addListener: (
    eventName: string,
    listener: (...args: any[]) => void
  ) => void
\end{minted}

Alias for \code{channel.on(string, ...)}.

% -------------------- %

\paragraph{channel.removeListener(...)}

\begin{minted}{typescript}
  removeListener: (
    eventName: string,
    listener: (...args: any[]) => void
  ) => void
\end{minted}

Removes the specified \code{listener} from the listener array for the specified \code{eventName}.

% -------------------- %

\paragraph{channel.removeAllListeners(...)}

\begin{minted}{typescript}
  removeAllListeners: (
    eventName?: string
  ) => void
\end{minted}

Removes all listeners, or those of the specified \code{eventName}.

\begin{itemize}
  \setlength\itemsep{-0.8em}
  \item \code{eventName}: The name of the event all listeners will be removed from.
\end{itemize}

% -------------------- %

\printfn
