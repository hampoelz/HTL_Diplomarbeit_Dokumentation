\addurlfn{gitignore}{\code{@github/gitignore}/Node.gitignore}{https://github.com/github/gitignore/blob/main/Node.gitignore}
\addurlfn{capacitor-electron}{\code{@capacitor-community/electron}}{https://github.com/capacitor-community/electron}

\addurlfn{vite-web}{Vite}{https://vitejs.dev}
\addurlfn{rollup-web}{Rollup.js}{https://rollupjs.org}

\addurlfn{rollup}{Rollup}{https://npmjs.com/package/rollup}
\addurlfn{rollup-commonjs}{\enquote{commonjs}}{https://npmjs.com/package/@rollup/plugin-commonjs}
\addurlfn{rollup-node-resolve}{\enquote{node-resolve}}{https://npmjs.com/package/@rollup/plugin-node-resolve}
\addurlfn{rollup-json}{\enquote{json}}{https://npmjs.com/package/@rollup/plugin-json}


\subsubsection{Complex Projects}
\label{sec:Capacitor-NodeJS:ComplexProjects}

The examples in this guide are a continuation of the examples in the \hyperref[sec:Capacitor-NodeJS:GettingStarted]{Getting Started} guide.

\paragraph{Custom starting point}
\label{sec:Capacitor-NodeJS:CustomStartingPoint}

In the \hyperref[sec:Capacitor-NodeJS:GettingStarted]{Getting Started} guide, the default starting point \code{index.js} was used for the Node.js project.
However, the main script can be renamed or moved to subdirectories for a better organized project.

To change this starting point, add a file called \code{package.json} to the Node.js project, which describes the project more in detail.
Using the \code{main} field in this file, a custom starting point for the Node.js project can be specified.
This should be a module relative to the root of the Node.js project directory. \cite{npm}

The package.json file could look like the following, if the \code{main} field is set to \code{server.js}:

\begin{minted}{javascript}
// static/nodejs/package.json
{
  "name": "capacitor-nodejs-project",
  "version": "1.0.0",
  "main": "./server.js"
}
\end{minted}

The project structure should then change to something like this:

\begin{minted}{diff}
  capacitor-app/
  ├── ...
  ├── dist/
  ├── src/
  ├── static/
  │   ├── nodejs/             # Node.js project directory
- │   │   ├── index.js        # main script (old)
+ │   │   ├── server.js       # main script (new)
+ │   │   ├── package.json    # starting point
  ├── capacitor.config.json
  ├── vite.config.ts
  ├── ...
\end{minted}

\newpage

\paragraph{Install Node.js Modules}
\label{sec:Capacitor-NodeJS:InstallModules}

To install Node.js modules, the project requires a \code{package.json} file.~\cite{npm}
See section \hyperref[sec:Capacitor-NodeJS:CustomStartingPoint]{Custom starting point} for more details.

The modules have to be installed in the Node.js project directory in which the \code{package.json} file was created using the npm CLI.
After installing modules, rebuild and sync the Capacitor project to update the application with the Node.js project.

For convenience, a postinstall script can be added to the main \code{package.json} in the root of the Capacitor project to automatically install the modules of the Node.js project \cite{npm}:

\begin{minted}{javascript}
// package.json
{
  "scripts": {
    "postinstall": "cd static/nodejs/ && npm install"
  },
  // other config options
}
\end{minted}

\begin{quote}
  You may also want to add a gitignore file to ignore unnecessary files.
  To do this, create a new file called \code{.gitignore} in the Node.js project directory and copy the contents of \fn{gitignore} into it.
\end{quote}

\begin{important}[Important]
  If the \fn{capacitor-electron} plugin is used, packaging with the electron-builder may cause problems since it does not include the modules installed in the Node.js project by default. \cite{electron-builder}
  \\[1em]
  To fix this issue, add the configuration \code[javascript]{"includeSubNodeModules": true} to the \code{electron-builder.config.json}.
  \cite{electron-builder}
\end{important}

\newpage

\paragraph{Improve Node.js loading times}
\label{sec:Capacitor-NodeJS:ImproveLoadingTimes}

The Node.js project can quickly grow very large when installing modules.
For projects that contain a large number of files, the load time can be reduced by decreasing the number of files and the file sizes.
\cite{nodejs-mobile:docs}

For this reason, it is recommended to use bunder tools such as \fn{rollup-web}.
In the following example, \fn{rollup-web} is used to bundle the Node.js project with all its modules to a single file.
\cite{rollup, rollup-plugins}

To get started install \fn{rollup} and its plugins \fn{rollup-commonjs}, \fn{rollup-node-resolve} and \fn{rollup-json} into the root of the Capacitor project.
If \fn{vite-web} is used as build system, \fn{rollup} is already pre-installed and does not need to be installed:

\begin{minted}{bash}
# Install Rollup (If Vite is used, this command is not needed)
npm i --save-dev rollup

# Install Rollup Plugins
npm i --save-dev @rollup/plugin-commonjs
npm i --save-dev @rollup/plugin-json
npm i --save-dev @rollup/plugin-node-resolve
\end{minted}

Since the Node.js project is now to be bundled, the project structure needs some changes.
The Node.js project should no longer be copied directly from \fn{vite-web} to the Capacitor webDir directory, instead it will be bundled with \fn{rollup}.

This means that the Node.js project directory needs to be moved from the static assets to somewhere else.
For example to the root directory of the Capcitor project:

\begin{minted}{diff}
  capacitor-app/
  ├── ...
  ├── dist/
  ├── src/
  ├── static/
- │   ├── nodejs/
- │   │   ├── node_modules/
- │   │   ├── server.js
- │   │   ├── package.json
- │   │   ├── ...
+ ├── nodejs/
+ │   ├── node_modules/
+ │   ├── server.js
+ │   ├── package.json
+ │   ├── ...
  ├── capacitor.config.json
  ├── vite.config.ts
  ├── ... 
\end{minted}

\begin{quote}
  Don't forget to update the new path to the project in the postinstall script,
  if one is used, as described in the \hyperref[sec:Capacitor-NodeJS:InstallModules]{Installing Node.js modules} section. 
\end{quote}

\newpage

After the restructuring of the project, \fn{rollup} can be configured.
Create a new file called \code{rollup.config.mjs} with the following content \cite{rollup, rollup-plugins}:

\begin{minted}{typescript}
// rollup.config.mjs
import commonjs from '@rollup/plugin-commonjs';
import json from '@rollup/plugin-json';
import nodeResolve from '@rollup/plugin-node-resolve';

export default {
  input: 'nodejs/server.js',
  output: {
    file: 'dist/nodejs/index.js',
    format: 'cjs',
  },
  external: ['bridge'],
  plugins: [
    commonjs(),
    json(),
    nodeResolve({
      preferBuiltins: true,
    }),
  ],
}; 
\end{minted}

To add bundling of the Node.js project to the build steps, modify the main \code{package.json} in the root of the Capacitor project 
and add \code[bash]{&& rollup -c rollup.config.mjs} to the \code{build} entry in the \code{scripts} object \cite{npm, rollup}:

\begin{minted}{diff}
# package.json
{
  "scripts": {
-   "build": "vite build"
+   "build": "vite build && rollup -c rollup.config.mjs"
  }
}
\end{minted}

So the project structure should look something like this:

\begin{minted}{diff}
  capacitor-app/
  ├── ...
  ├── dist/
  ├── src/
  ├── nodejs/
  │   ├── node_modules/
  │   ├── server.js
  │   ├── package.json
  │   ├── ...
  ├── capacitor.config.json
+ ├── rollup.config.mjs
  ├── vite.config.ts
  ├── ... 
\end{minted}

After building and syncing the project, the Node.js runtime should start faster now.

\newpage

\paragraph{Manual Node.js runtime start}
\label{sec:Capacitor-NodeJS:ManualRuntimeStart}

By default, the Node.js runtime starts automatically with application start.
However, this behavior may not be suitable for all projects. 

This behavior can be disabled globally via the \code{startMode} plugin configuration:

\begin{minted}{diff}
# in capacitor.config.json or capacitor.config.ts
{
  "webDir": 'dist',
  "plugins": {
    "CapacitorNodeJS": {
      "nodeDir": "nodejs",
+     "startMode": "manual",
    },
  },
} 
\end{minted}

Now the Node.js runtime has to be started manually with the \code{NodeJS.start()} command:

\begin{minted}{typescript}
import { NodeJS } from 'capacitor-nodejs';

// Starts the Node.js engine.
NodeJS.start();
\end{minted}

Manually starting the Node.js runtime provides options to override the \code{nodeDir} configuration or even the path for the main script.

In addition, arguments can be passed to the main script and environment variables for the Node.js runtime can be set:

\begin{minted}{typescript}
import { NodeJS } from 'capacitor-nodejs';

// Options for starting the Node.js engine manually.
const options = {
  args: [ "--option", "value" ],
  env: {
    "DB_HOST": "localhost",
    "DB_USER": "myuser",
    "DB_PASS": "mypassword"
  }
}

// Starts the Node.js engine with properties as set by the `options`.
NodeJS.start(options);
\end{minted}

\newpage

\paragraph{Data storage}
\label{sec:Capacitor-NodeJS:DataStorage}

Mobile platforms are different than the usual desktop platforms in that they require applications to write in specific sandboxed paths and don't have permissions to write elsewhere.
\cite{nodejs-mobile:docs}

The built-in bridge module provides an \ac{api} to get a per-user application data directory on each platform:

\begin{minted}{javascript}
const { getDataPath } = require('bridge');

// Get a path where data can be read and written.
const dataPath = getDataPath();
\end{minted}

\begin{warning}[Warning]
  Do not use the Node.js project directory itself for data storage, it will be overwritten after each application update!
  \cite{nodejs-mobile:docs}
\end{warning}

To get a path for temporary files, the node.js inbuilt method \code{os.tmpdir()} can be used \cite{nodejs}:

\begin{minted}{javascript}
const os = require('os');

// Get a path for temporary files.
const tmpPath = os.tmpdir();
\end{minted}

\begin{warning}[Warning]
  On Android, the files in the cache are kept until the system needs space, so it increases the application's disk space unless the developer manually deletes them.
  \cite{nodejs-mobile:docs}
\end{warning}

\printfn
