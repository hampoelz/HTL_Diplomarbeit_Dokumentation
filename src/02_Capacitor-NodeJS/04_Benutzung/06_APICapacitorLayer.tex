\subsubsection{API - Capacitor layer}
\label{sec:Capacitor-NodeJS:API_CapacitorLayer}

The \code{NodeJS} module is the \ac{api} used in the Capacitor app.
It provides a few methods to send messages from the Node.js layer and wait for them.

It has the following methods:

\begin{itemize}
  \setlength\itemsep{-0.8em}
  \item \code{start(...)}
  \item \code{send(...)}
  \item \code{whenReady()}
  \item \code{addListener(string, ...)}
  \item \code{removeListener(...)}
  \item \code{removeAllListeners(...)}
  \item Interfaces
  \item Type Aliases
\end{itemize}

% -------------------- %

\paragraph{start(...)}

\begin{minted}{typescript}
  start(
    options?: StartOptions
  ) => Promise<void>
\end{minted}

Starts the Node.js engine with properties as set by the \code{options}.

\textbf{Note:} This method is only available if the Node.js engine startup mode was set to \code[typescript]{'manual'} via the plugin configuration.

% -------------------- %

\paragraph{send(...)}

\begin{minted}{typescript}
  send(
    args: ChannelPayloadData
  ) => Promise<void>
\end{minted}

Sends a message to the Node.js process.

% -------------------- %

\paragraph{whenReady()}

\begin{minted}{typescript}
  whenReady() => Promise<void>
\end{minted}

Resolves when the Node.js process is initialized.

% -------------------- %

\newpage

\paragraph{addListener(string, ...)}

\begin{minted}{typescript}
  addListener(
    eventName: string,
    listenerFunc: ChannelListenerCallback
  ) => Promise<PluginListenerHandle> & PluginListenerHandle
\end{minted}

Listens to \code{eventName} and calls \code{listenerFunc(data)} when a new message arrives from the Node.js process.

\textbf{Note:} When using the Electron platform, \code{PluginListenerHandle.remove()} does not work due to limitations.~\cite{capacitor-electron}
Use \code{removeListener(listenerFunc)} instead.

\textbf{Returns:} \code[typescript]{Promise<PluginListenerHandle> & PluginListenerHandle}

% -------------------- %

\paragraph{removeListener(...)}

\begin{minted}{typescript}
  removeListener(
    listenerHandle: PluginListenerHandle
  ) => Promise<void>
\end{minted}

Removes the specified \code{listenerHandle} from the listener array for the event it refers to.

% -------------------- %

\paragraph{removeAllListeners(...)}

\begin{minted}{typescript}
  removeAllListeners(
    eventName?: string
  ) => Promise<void>
\end{minted}

Removes all listeners, or those of the specified \code{eventName}, for this plugin.

% -------------------- %

\newpage

\paragraph{Interfaces}


\subparagraph{StartOptions}

An interface containing the options used when starting the Node.js engine manually.

\begin{interfacedesc}[Capacitor-NodeJS / StartOptions][5em]
  \code{nodeDir} & \code[typescript]{string}   & Relative path of the integrated Node.js project based on the Capacitor webdir. Defaults to the \code{nodeDir} field of the global plugin configuration. If the \code{nodeDir} config is not set, \code{nodejs} in the Capacitor webdir is used as Node.js project directory. \\ \hline
  \code{script}  & \code[typescript]{string}   & The primary entry point to the Node.js program. This should be a module relative to the root of the Node.js project folder. Defaults to the \code{main} field in the project's package.json. If the \code{main} field is not set, \code{index.js} in the project's root folder is used. \\ \hline
  \code{args}    & \code[typescript]{string[]} & A list of string arguments. \\ \hline
  \code{env}     & \code[typescript]{NodeEnv}  & Environment key-value pairs. \\ \hline
\end{interfacedesc}

% -------------------- %

\subparagraph{NodeEnv}

An interface that holds environment variables as string key-value pairs.

% -------------------- %

\subparagraph{ChannelPayloadData}

The payload data to send a message to the web page via \code{eventName},
along with arguments. Arguments will be serialized with \ac{json}.

\begin{interfacedesc}[Capacitor-NodeJS / ChannelPayloadData]
  \code{eventName} & \code[typescript]{string} & The name of the event being send to. \\ \hline
  \code{args}      & \code[typescript]{any[]}  & The array of arguments to send. \\ \hline
\end{interfacedesc}   

% -------------------- %

\subparagraph{PluginListenerHandle}

\begin{interface}[Capacitor-NodeJS / PluginListenerHandle]
  \code{remove} & \code[typescript]{() => Promise<void>} \\ \hline
\end{interface}

% -------------------- %

\subparagraph{ChannelCallbackData}

The callback data object when a message from the Node.js process arrives.

\begin{interfacedesc}[Capacitor-NodeJS / ChannelCallbackData]
  \code{args} & \code[typescript]{any[]} & The received array of arguments. \\ \hline
\end{interfacedesc}

% -------------------- %

\newpage

\paragraph{Type Aliases}


\subparagraph{ChannelListenerCallback}

The callback function to be called when listen to messages from the Node.js process.

\code[typescript]{(data: ChannelCallbackData): void}
