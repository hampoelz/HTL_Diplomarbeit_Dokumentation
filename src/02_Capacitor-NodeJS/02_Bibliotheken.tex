\subsection{Verwendete Bibliotheken}

Die Node.js Runtime wird zum Zeitpunkt dieser Arbeit offiziell nur auf gängigen Desktop"=Plattformen (Linux, Windows und macOS) unterstützt.~\cite{nodejs}
Daher musste für die Implementierung des Plugins zusätzlich zu den Plattform-\acsp{api} auf weitere Bibliotheken und Toolsets zurückgegriffen werden.

In diesem Abschnitt werden die Bibliotheken und Toolsets vorgestellt, die bei der Entwicklung des Capacitor-NodeJS Plugins verwendet wurden.

\subsubsection{Node.js for Mobile Apps}

Node.js for Mobile Apps ist ein Toolset für die Integration der Node.js Runtime in Mobile"=Anwendungen.
Das Hauptziel des Projekts ist es, Korrekturen sowie eine Bibliothek bereitzustellen, damit Node.js auch auf Mobile"=Plattformen (Android und iOS) problemlos ausgeführt werden kann.
\cite{nodejs-mobile, nodejs-mobile:docs}

Im Capacitor-NodeJS Plugin werden vom Projekt bereitgestellte Header-Dateien der mobilen Node.js Runtime sowie vorgefertigte native Binärdateien für Android und iOS verwendet.

\subsubsection{Android NDK}

Android \ac{ndk} ist ein Toolset, mit dem Teile einer Android Anwendung in nativem Code mit Sprachen wie C und C++ implementiert werden können.
Es bietet eine Reihe von Bibliotheken und Tools, die zur Erstellung von nativem Code für Android verwendet werden können.
\cite{android:ndk}

Damit Android Anwendungen nativen C oder C++ Code aufrufen können und umgekehrt, verwendet das Android \ac{ndk} das \ac{jni} Framework.
Dies bedeutet, dass Java-Code geschrieben werden kann, der auf Low"=Level"=Hardwarefunktionen zugreifen oder Funktionen in einer nativen Bibliothek aufrufen kann.
\cite{android:ndk}

Im Capacitor-NodeJS Plugin wird das Android \ac{ndk} benötigt, um auf die native JavaScript Runtime des Node.js for Mobile Apps Toolkits zuzugreifen.
\cite{nodejs-mobile:docs}
