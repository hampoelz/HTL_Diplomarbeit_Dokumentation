\addurlfn{nodejs-mobile}{Node.js for Mobile Apps}{https://github.com/nodejs-mobile/nodejs-mobile}
\addurlfn{android-ndk}{Android \ac{ndk}}{https://developer.android.com/ndk}

\subsection{Verwendete Bibliotheken}

Die Node.js Runtime wird zum Zeitpunkt dieser Arbeit offiziell nur auf gängigen Desktop"=Plattformen (Linux, Windows und macOS) unterstützt.~\cite{nodejs}
Daher musste für die Implementierung des Plugins zusätzlich zu den Plattform-\acsp{api} auf weitere Bibliotheken und Toolsets zurückgegriffen werden.

In diesem Abschnitt werden die Bibliotheken und Toolsets vorgestellt, die bei der Entwicklung des Capacitor-NodeJS Plugins verwendet wurden.

\subsubsection{Node.js for Mobile Apps}

\fn{nodejs-mobile} ist ein Toolset für die Integration der Node.js Runtime in mobilen Anwendungen.
Das Hauptziel des Projekts ist es, Korrekturen bereitzustellen, damit Node.js auch auf mobilen Plattformen problemlos ausgeführt werden kann.
Eine Hauptkomponente des Projekts besteht aus einer Bibliothek, die es ermöglicht, Node.js auf einem separaten Thread in mobilen Anwendung auszuführen.
\cite{nodejs-mobile, nodejs-mobile:docs}

Im Capacitor-NodeJS Plugin werden vom Projekt bereitgestellte Header-Dateien der mobilen Node.js Runtime sowie vorgefertigte native Binärdateien für Android und iOS verwendet.

\subsubsection{Android NDK}

Das \fn{android-ndk} ist ein Toolset, mit dem Teile einer Android Anwendung in nativem Code mit Sprachen wie C und C++ implementiert werden können.
Es bietet eine Reihe von Bibliotheken und Tools, die zur Erstellung von nativem Code für Android verwendet werden können.
\cite{android:ndk}

Damit Android Anwendungen nativen C oder C++ Code aufrufen können und umgekehrt, verwendet das \fn{android-ndk} das \ac{jni} Framework.
Dies bedeutet, dass Java-Code geschrieben werden kann, der auf Low"=Level"=Hardwarefunktionen zugreifen oder Funktionen in einer nativen Bibliothek aufrufen kann.
\cite{android:ndk}

Im Capacitor-NodeJS Plugin wird das \fn{android-ndk} benötigt, um auf die native JavaScript Runtime des \fn{nodejs-mobile} Toolkits zuzugreifen.
\cite{nodejs-mobile:docs}

\printfn
